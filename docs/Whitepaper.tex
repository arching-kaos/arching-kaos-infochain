\documentclass[10pt,a4paper,twocolumn]{paper}
\usepackage[utf8]{inputenc}
\usepackage[T1]{fontenc}
\usepackage[english]{babel}
\usepackage{amsmath}
\usepackage{amsfonts}
\usepackage{amssymb}
\usepackage{graphicx}
\usepackage{hyperref}
\title{Arching Kaos Infochain Whitepaper}
\author{Kaotisk Hund}
\begin{document}

	\maketitle
	\begin{abstract}
		
	\end{abstract}
	\tableofcontents
	\section{Introduction}\label{Introduction}
	The following is a document on how an info-chain is may be constructed and on how it could be analyzed. Along the way, Arching Kaos Info-chain is introduced as a proof of concept. The last decade, block chain technology covered many aspects of financial and data exchange requirements. Many applications have been developed based on crypto-currency networks. Along side, smart contracts are a part of the many possibilities someone can use. To make a chain, most of these technologies have a reference into every block of data to the previous valid block. This is the block-chain. Nodes of the network validate each block so there is one block-chain.


	\section{Arching Kaos}\label{arching-kaos}
	The words "Arching Kaos" are referring to the Asset\footnote{Asset is the name of a certain amount of tokens where holders of them can perform transactions.} used on Stellar Network, a decentralized\footnote{Decentralized is a cluster of computers working together in a network. They mostly with mesh topology which has no single point of failure.} block-chain\footnote{Block chain is a "chain" of blocks. Blocks mostly contain information. Chain is formed because every block commonly say which chain belongs to and which block was just before that. Hence, from a certain block we can retrieve the chain before it.}. There is a project called "Arching Kaos" which is issuing and distributing the Asset. It's main cause is to create a distributed and decentralized web-radio station.
	
	\section{Infochain}\label{infochain}
	The word "Infochain" is used as reference to the chain of information we are producing.
		
	Firstly, we build our "chain" on top of a block-chain. We use the transactions to pass information, so on top of a block and add information. Now, we already are in a meta-block-chain. For now, we are not validating transactions as the rest of block chain technologies do. We add the previous "block" but anyone can reference to that as well. Their inserts won't be discarded by anyone. Multiple types of blocks can co-exist as well. Anything can be "posted" on the chain. It can be a fork of it, a whole different chain, or a "main" place. People can use encryption as well and post encrypted messages as well.
	
	There is also the need to read these information. To do that, an open protocol has been developed, since the use of such a chain is possible to be used for far too many cases. Different or bundled application can be developed to read the chain and present it to the public through internet. To do that we create one protocol, a predefined set of rules on how the information will be formatted and structured.
	
	We make use of IPFS to store information in a way that we can retrieve them and validate them, serialize them and present them.
	
	IPFS is creating a CID for each file or folder is added to its repository. Furthermore, while connected to any network with other instances of IPFS running, the same information can be retrieved using the same CID.
	
	We want to be descriptive on what we want to do, while we maintain a series or chain of information.

	\section{Protocol}\label{protocol}
	To build a protocol we need to know what it's requirements are. By specifications of the memo text length on Stellar Network transactions, we know that the maximum length of an alphanumerical string is 28 bytes. We also know from specifications of IPFS that its CIDv0 is 46 bytes. Our first concern is to publish the IPFS CID to the Stellar network.
	
	How things need to be done in a way that works and is reliable in the sense of what we want to achieve. In particular, what we do is encode an IPFS CID, split it and add its sliced parts into memo text in transactions of an asset \ref{asset} on Stellar Network.
	

	\subsection{Minimum transactions}\label{minimum_transactions}
	To send the CID (46bytes) using a memo (28bytes),we need 2 transactions, this means that we can use the memo field twice ($ 28bytes*2transactions=56bytes $). Now, the CID $ 46bytes/2parts=23bytes $ fits into the memos. There is a difference though $ 28bytes-23bytes=5bytes $ and a total of $56bytes-46bytes=10bytes$. These 10 bytes will be used to fill the gaps in a way that is recognizable to help us make assumptions about each transaction's memo later.
	
	\subsection{Encoding}\label{encoding}
	Since we have 10bytes more, we can make use of them in a way that we can tell which part of the CID we want, appears in the transaction memo.
	
	We split the 46bytes string into 11,12,12 and 11bytes\\
	$ CID_{a} = CID[1-11] $\\
	$ CID_{b} = CID[12-23] $\\
	$ CID_{c} = CID[24-35] $\\
	$ CID_{d} = CID[36-46] $\\in sequence for CID.
	
	Then, we append or prep-end the fitting amount of dots in each part to reach the maximum of 14bytes in order to reach maximum length of the transaction memo, so $DOTS_{3}$ and $DOTS_{2}$. For the first 2 parts we add prefixes and for the last ones we append:\\
		$ Memo_{PartA} = DOTS_{3} + CID_{a} $\\
		$ Memo_{PartB} = DOTS_{2} + CID_{b} $\\
		$ Memo_{PartC} = CID_{c} + DOTS_{2} $\\
		$ Memo_{PartD} = CID_{d} + DOTS_{3} $

	\subsection{Referencing}\label{referencing}
	We also want to make sure that what we read is unique. By sending 2 transactions the CID can not be retrieved in sequence. We have a first part and a second one. Maybe congestion could result to many first parts in sequence and second ones mixed later in the chain.
	
	In order to achieve continuity, we need to reference in each transaction parts from other memos.
	
	
	So, we split the transaction memo in two parts ($ 28bytes/2parts=14bytes $). In two transactions we can have 4 parts. Each part can be used as a reference to find the rest of the parts.
	
	\subsection{Multiplexing the CID}\label{multiplexing}
	The CID can fit in that 56byte string provided by the minimum length ($ 56>46 $). There is a difference though $ 56bytes-46bytes=10bytes $ which we can use to make some formatting. Also, 56 can be divided with 14 to give as 4 parts. Then can use each part of it, add another next to it and send it as transaction. One way to do is for parts A, B, C and D: AB, AC, AD where here is the original message ABCD. We make use of the 10bytes filled with dots, making unique additions for every part, instead of leaving blank the last 10 bytes of the D part. So we do as follows, from CID:


	We have 4 parts of 14bytes. We can now mix them like this:\\
	$Transaction_{1} = Memo_{PartA}+Memo_{PartB}$\\
	$Transaction_{2} = Memo_{PartA}+Memo_{PartC}$\\
	$Transaction_{3} = Memo_{PartA}+Memo_{PartD}$

	Because of splitting a unique message doesn't necessarily provide unique messages. Two or more CIDs can share some parts of the CID. To overcome this, we add two more parts, referencing B part this time, including the parts that been referenced only once ($Memo_{PartC}$, $Memo_{PartD}$):\\
	$Transaction_{4} = Memo_{PartB}+Memo_{PartC}$\\
	$Transaction_{5} = Memo_{PartB}+Memo_{PartD}$
	
	In total now we have 5 memos to send. Because of what we did, splitting 1 IPFS CID into 5 transaction memos, we will send them by moving 0.2 of the asset for each transaction while for all of them it would be paid 1 asset unit. In this case, we assume that this one pair of transactions made the containing message (CID) an asset called ZBLOCK\ref{zblock}. However, payments can vary, which means that different amounts can be decoded differently as well.
	
	
	\subsection{Decoding}\label{decoding}
	
	To decode, we gather transaction memos and start scanning for possible ZBLOCK CIDs encoded in the way described above.
	
	If we find a similar formatted transaction memo matching either AB, AC, AD, BC or BD memos expected, we try to find the rest parts in the transactions. After that we have a list of IPFS CIDs that can be further analyzed.
	
	We can now say that we added the CID, but what would be a useful way to add into IPFS that we can them information further more? We need a structure.
	
	\section{Structure}\label{structure}

	In order to be prepared for the chaos of information it can be dumped into the chain in any case, certain requirements can be accomplished according to the needs that can be raised up. So we should specify a prototype of how things look like in the parts we are going to read.
	
	Two procedures are coming into place. We need to craft messages and we need to read them. Encapsulation of information is used.

	Our first concern is to publish useful information that can be decoded easily. The prototype is a way to make an example usage.
	
	CIDs published in the asset's transactions should contain JSON formatted messages. We will refer to this as ZBLOCK.
	
	\subsection{ZBLOCK}\label{zblock}
	This IPFS CID is called ZBLOCK (z block \textless- ze block \textless-
	the block).
	It should be a JSON object with two IPFS CIDs. ZBLOCK consists of a BLOCK and its detached signature in JSON format.
	

	\subsubsection{BLOCK}\label{block}
	
	A BLOCK contains action, data, detach signature of data, previous CID and public GPG key of the publisher.

	\paragraph{ACTION}\label{action}
	An action.
	
	
	Can be literally anything! A convinient way to use it is refering to an
	object or a subject and to a verb using URL like scheme: e.g.:
	mixtape/add
	
	\paragraph{DATA}\label{data}
	A data JSON object relative to the action.
	
	
	DATA should be the place to be creative. It's rendered based on the action that is mentioned in the BLOCK. Can be literally anything, but is specified by the ACTION's handler. We use IPFS CID to reference to it.
	
	\paragraph{DETACH}\label{detach}
		The detached signature of the DATA JSON object.
		
		GPG is used to sign crypto-graphically the DATA JSON file. We added directly, not through IPFS. Although we can do that as well.
	\paragraph{KEY}\label{key}
	The GPG key of the signer/contributor of both the block and data.
	 
	 
	It should be the same to be very sure about who contributed. There is no implementation of the checking for now since, but the tool to add new blocks require it.
	 
	\paragraph{PREVIOUS}\label{previous}
	The previous ZBLOCK.
	 
	 
	This is done if we want to go back to previous transactions but we don't have all the transactions. If the chain is maintained we can walk back to it through IPFS with out the need of reading more transactions from Stellar network.
	
	
	\subsection{BLOCK\_SIGNATURE}\label{block-signature}
	This is the IPFS CID of a detached PGP/GPG signature of the BLOCK.
	
	
	\section{Asset}\label{asset}
	An asset, created on Stellar Network, is a representative model of a stock in stock market. Stock market is including companies, goods and maybe other types. We are using it as a utility. Our asset's usage is the ability to make use of an infochain using a protocol. 1 asset equals the symbolic contribution to the network.

	\section{Tokenomics}\label{tokenomics}
	See \ref{asset-distribution}
	
	The way that this system is designed, is to get into trusted hands that would not disrupt its problems. We will be investigating the testnet for possible errors. To avoid incoming funds to the distribution account or some specific route, all the tokens should go to the market place.

	\section{Benefits holding tokens}
	Holding tokens gives the holder the opportunity to participate in the infochain. In order to participate, someone needs to pay these tokens to other(s). The minimum of a button sending thing is mark as a `transaction\_number`. This means that the sender, approved or shared or like or found somewhat this transaction to be important. You can send then 5 transactions, 0.2 assets each to send something either to yourself either to someone else. You can talk with people, sending them 1 or more transactions. At least 0.2 of your holding assets will move to the destination. This is because when we speak to someone, we are moving energy to them.

	You can also mark your lost profile. If people know you, they post an acceptance of the change message. If you found your keys, you can "delete" the event. In case your data was stolen from you and the thief tries to reuse your account (like sending a "reject message" to your new id), then you can continue from there, re-establishing your previous profile and make sure your friends follow you instead of the compromised account. If such thing happens, you will practically announce to the rest of surfers that there is a shark over there.

	Attempts for impersonation can do the same technic. But if your friends stand with you a diregard your ID thief then you win the argument. It's like voting but instead of voting you actually do something in order to progress yourself and overcome the ID theft. However, you should have in mind, that thieves can get all the data stored into your account.
	
	So, one thing to remember is that: "Always keep private the information you don't want to be published". There is no other way to prevent such events. Do not store your real ID, passport, driving license or other information that can be used in any way by third parties.
	
	\section{Ownership}\label{ownership}
	
	There is no direct ownership, the database that is created in this system can be attached to the transaction sender or the GPG key of them. However, the correct word is "contributor". The contributor is the Stellar address, packed with the GPG key. The first contributor is supposed to be the original poster or content creator. Since we can not guarantee that everyone is who they tell, we can do a voting to assign the rightful "owner" or "creator" of the piece of information.
	
	\section{Security}\label{security}
	There is no guarantee for anything. You can protect yourself only with encrypting the data you want to send. This is easy to do. You can also publish IPFS links that are part of a closed swarm. It's a different base key instead of the public one. What you achieve is that noone will dirupt your database also, but you need to adjust also in the settings of the renderers and publish them.

	Of course, such a procedure has it's complexity. The best way to do it is with creating a new asset and start directly there.


	\section{Privacy}\label{privacy}

	Your data is yours but whatever you publish unencrypted, is everyones'. It's public information, that cannot be undone. Anyone can use it to refer to you.

	In order to avoid your sensitive information get stolen, you should use GPG/PGP encryption, or whichever other you trust or you could also reconsider if you want this piece of information publicly available. If you post something by accident, there is NO way to delete it.

	Me, Kaotisk Hund, as the internet figure or the real-life caster of this system, I am responsible for bringing this information and tools to you. 
	
	You and every user of the algorithm is responsible of how they use it.

	Malicious software, illegal activities, copy-writing violations are NOT welcome. But there is no guarantee that all the users or some of them will not break this.

	Each user can have a 5-key identity containing: \begin{itemize}
	\item their Stellar address,
	\item their IPFS ID,
	\item their SSB ID (if there is),
	\item their CJDNS IP (if any) and
	\item their public GPG KEY.
	\end{itemize}. Whatever someone puts in the infochain, their fingerprint is everywhere.
	
	The basic requirements are the Stellar address and IPFS in order to run the algorithm in order to add.
	
	\section{Encryption}
	GPG is used for creating detached signatures. You can skip that or go full encryption by encrypting the whole contribution of yours or parts of it.
	
	To manage a monetizing place (stellar address) with a file-hosting service (IPFS id) with a social profile (SSB id), a networking address for hosting (CJDNS ip) and an encryption/signing key (GPG key).

	Possibly, someone can generate everything, retry stuff and try attack the network. Or simply try to make a half thing to get hands on the funds as it's been proven to be a method of attack on assets and tokens on other block-chain networks.

	I don't know how to prevent that, or what should I do to keep everyone off the funds. \ref{asset-distribution}.
\section{Appendix}\label{appendix}
\subsection{Testing}\label{testing}

Someone needs an amount of asset to make a transaction. Read
\ref{asset-distribution} for info on this.

During the development and research common tools were used. Testing was done on Fedora Linux 34 workstation environment. The tools were run using:
\begin{enumerate}
	\item Python
	\item Bash
	\item Javascript
	\item Firefox
\end{enumerate}

For Python, the installation on a local environment of stellar-sdk was used.

\section{Examples}\label{examples}

\subsection{Encoding of IPFS CID}\label{example-encoding}

We have the following IPFS CID of an empty file (is broken with spaces to fit in the document):\\
QmbFMke1KXq\\
nYyBBWxB74N4\\
c5SBnJMVAiMN\\
RcGu6x1AwQH\\

We mix it with dots in this way:
...QmbFMke1KXq\\
..nYyBBWxB74N4\\
c5SBnJMVAiMN..\\
RcGu6x1AwQH...\\

\subsection{Split encoded text to 4 parts}\label{example-split}

\begin{enumerate}
	\item[A] ...QmbFMke1KXq
	\item[B] ..nYyBBWxB74N4
	\item[C] c5SBnJMVAiMN..
	\item[D] RcGu6x1AwQH...
\end{enumerate}



\subsection{Mix its parts}\label{example-multiplex}
\begin{enumerate}
	\item[AB] ...QmbFMke1KXq..nYyBBWxB74N4
	\item[AC] ...QmbFMke1KXqc5SBnJMVAiMN.. 
	\item[AD] ...QmbFMke1KXqRcGu6x1AwQH...
	\item[BC] ..nYyBBWxB74N4c5SBnJMVAiMN..
	\item[BD] ..nYyBBWxB74N4RcGu6x1AwQH...
\end{enumerate}

\subsection{Example BLOCK}\label{example-block}
\begin{verbatim}
"BLOCK":{
    "action":"something/do",
    "data":"DATA_JSON_OBJECT_IPFS_CID",
    "detached":"INLINE_DETACHED_SIGNATURE_OF_DATA",
    "key":"GPG_PUBLIC_KEY",
    "previous":"PREVIOUS_ZBLOCK"
}
\end{verbatim}


	
\section{Notes}\label{notes}

	\subsection{Asset Distribution}\label{asset-distribution}

Started with issuing 10,000 ARCHINGKAOS.

Current price for one (1) transfer and store of information, using
Arching Kaos Infochain is 1 ARCHINGKAOS.

Supposely, after someone acquires 1 ARCHINGKAOS, it's possible to reuse
it by sending either to theirselves, or recycling it on their wallets,
while transfering and posting content.

Theoritically, when someone learns something, they don't forget and they
can do the learnt thing, again and again.

Because of that and due to the trust issues that governing our
relations, the distribution of the asset is something very important.

The ultimate goal here is for everyone to be able to trust each other.
That would mean that on earth, peace have been achieved and actions of
people are based on their hearts with compassion, intuition and love.

The point is not to just give away to everyone, neither to restrict
within a closed community of the ``awesome''. The point is to maintain a
caring community and expand it. Sky is the limit, maybe there is no
limit at all!

So, since 2019, the start of the project over the internet called
Arching Kaos, there were and some still are some contributors that put
trust in the project along its subprojects: The contributors! Each one
of them, will get 1 ARCHINGKAOS to start with.

After a holder makes use of it, they can bring people into the network,
having also the responsibility of who they bring in.

	\paragraph{Previous thoughts}\label{previous-thoughts}

Another way of managing ARCHINGKAOS would be from a central automaton
that could create a ``bonding'' BLOCK that could much someone's Stellar
address with one generated and managed from the automaton. Think it like
this: when we touch, we don't really touch. So when you are getting
ARCHINGKAOS, you are NOT getting ARCHINGKAOS. Someone else is holding
them for you. It is like a lottery on a phone game. But instead of
getting to play or get paid and go, you just ask for some when you need
them to package your stuff.


	\paragraph{Further reading}\label{further-reading}

Since it's possible for everyone to reproduce this system into other
assets, I guess doing so is the best way to expand the system. Read more
on \ref{hacking}.

	\subsection{Invites}\label{invites}

Since we are building a interconnected Radio station, we need to get
more people in this.

We invite people by sending them ARCHINGKAOS. Hitting an invite button
and add them with email address or stellar address or something.

Note: You need to know who you are inviting, personally.


	\subsection{Handling}\label{handling}


\begin{verbatim}
"action":"user/add"
\end{verbatim}

the transactions address to the one we create 
\begin{verbatim}
{
    "ipfs.gpg",
    "ipfs.signature"
}
\end{verbatim}

We reply to by creating an address for them:
\begin{verbatim}
{
    "type":"ak/user/add",
    "akuser":"address that requested to register"
}
\end{verbatim}

 \begin{verbatim} 	
{
    "action":"(subject||object)/verb",
    "data":"ipfs.data.json",
    "detach":"detach-ipfs.data.json",
    "key":"ipfs.gpg.public.key",
    "previous":"ipfs.gpg.signature of the previous object"
}
\end{verbatim}

	\subsection{Hacking}\label{hacking}

If you want to hack this you are more than welcome to do so. Instead of only trying to break though, you can contribute your solution to the problem.

You can use this piece of software to maintain your own asset/token.

It is possible for someone to copy this program and use it to roll out another infochain on a different asset.

Things that would matter of research based on my opinion are the following. You can also make a proposal.

\subsubsection{Render all transactions and decide how to perform each BLOCK's action}
	This would be nice to have and use in general, it's creative subject.
	
\subsubsection{Download latest transactions (like 5-100) and climb rest of the chain through BLOCK's previous IPFS CID}
 This would interesting on
	figuring out if and how each BLOCK inherits the previous CIDs
	correctly. Assuming that anyone could forge the previous' value it
	would be really valuable to verify in comparison with the whole of
	transaction history which is always the safest route.
	
\subsubsection{Create subchains}
 Since our genesis block is an empty file, it would be
	interesting to start chains with other values as well. For example, a
	file containing 0 or 1 or other integers, simple words and/or phrases.


Happy hacking


	\subsection{Contribute}\label{contribute}

I would really appreciate any comments, issues, pull requests or whatever in order to make this better.

Since you may have found this on github\footnote{\url{https://github.com/kaotisk-hund/arching-kaos-infochain}} there are tools to help each other. If you feel doing so, you are more than welcomed to do so.

I worked mostly on how to split and stuck back the messages, so the "core" is developed, though this is done in parts.

Read ROADMAP file to get an idea onto which directions the project is mostly to move towards. Note, that ROADMAP is also incomplete.

\subsection{Discuss}
You can discuss in Keybase where you can find \textit{arching\_kaos} team.
Channels there are \textit{\#general}, \textit{\#roadmap}, \textit{\#infochain}. There is also IRC
client ready to use at \url{https://irc.arching-kaos.net}.

\end{document}
